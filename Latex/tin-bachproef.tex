%%========================================================================
%% LaTeX sjabloon voor stage/projectrapport of bachelorproef
%%  HoGent Bedrijf en Organisatie
%%========================================================================

%%========================================================================
%% Preamble
%%========================================================================

\documentclass[pdftex,a4paper,12pt,twoside]{report}

% XXX: Let op: dit sjabloon is gemaakt om dubbelzijdig af te drukken
% Voor enkelzijdig, verwijder ``twoside'' hierboven.

%%---------- Extra functionaliteit ---------------------------------------

\usepackage[utf8]{inputenc}  % Accenten gebruiken in tekst (vb. é ipv \'e)
\usepackage{amsfonts}        % AMS math packages: extra wiskundige
\usepackage{amsmath}         %   symbolen (o.a. getallen-
\usepackage{amssymb}         %   verzamelingen N, R, Z, Q, etc.)
\usepackage[dutch]{babel}    % Taalinstellingen: woordsplitsingen,
                             %  commando's voor speciale karakters
                             %  ("dutch" voor NL)
\usepackage{eurosym}         % Euro-symbool €
\usepackage{geometry}
\usepackage{graphicx}        % Invoegen van tekeningen
\usepackage[pdftex,bookmarks=true]{hyperref}
                             % PDF krijgt klikbare links & verwijzingen,
                             %  inhoudstafel
\usepackage{listings}        % Broncode mooi opmaken
\usepackage{multirow}        % Tekst over verschillende cellen in tabellen
\usepackage{rotating}        % Tabellen en figuren roteren
\usepackage{natbib}          % Betere bibliografiestijlen
\usepackage{fancyhdr}        % Pagina-opmaak met hoofd- en voettekst

\usepackage[T1]{fontenc}     % Ivm lettertypes
\usepackage{lmodern}
\usepackage{textcomp}

\usepackage{lipsum}          % Voor vultekst (lorem ipsum)

%%---------- Layout ------------------------------------------------------

% hoofdingen, enz.
\pagestyle{fancy}
% enkel hoofdstuktitel in hoofding, geen sectietitel (vermijd overlap)
\renewcommand{\sectionmark}[1]{}

% lijn, wordt gebruikt in titelpagina
\newcommand{\HRule}{\rule{\linewidth}{0.5mm}}

% Leeg blad
\newcommand{\emptypage}{
\newpage
\thispagestyle{empty}
\mbox{}
\newpage
}

% Gebruik een schreefloos lettertype ipv het "oubollig" uitziende
% Computer Modern
\renewcommand{\familydefault}{\sfdefault}

% Commando voor invoegen Java-broncodebestanden (dank aan Niels Corneille)
% Gebruik: \codefragment{source/MijnKlasse.java}{Uitleg bij de code}
\newcommand{\codefragment}[2]{ \lstset{%
  language=java,
  breaklines=true,
  float=th,
  caption={#2},
  basicstyle=\scriptsize,
  frame=single,
  extendedchars=\true
}
\lstinputlisting{#1}}

%%---------- Documenteigenschappen ---------------------------------------
%% Vul dit aan met je eigen info:

% Je eigen naam
\newcommand{\student}{Mathio Roelandt}

% De naam van je lector, begeleider, promotor
\newcommand{\promotor}{Bert Van Vreckem}

% De naam van je co-promotor
\newcommand{\copromotor}{Joeri Van Steen}

% Indien je bachelorproef in opdracht van een bedrijf of organisatie
% geschreven is, geef je hier de naam.
\newcommand{\instelling}{---}

% De titel van het rapport/bachelorproef
\newcommand{\titel}{Migratie naar Angular 2.0}

% Datum van indienen
\newcommand{\datum}{}

% Faculteit
\newcommand{\faculteit}{Faculteit Bedrijf en Organisatie}

% Soort rapport
\newcommand{\rapporttype}{Scriptie voorgedragen tot het bekomen van de graad van\\Bachelor in de toegepaste informatica}

% Academiejaar
\newcommand{\academiejaar}{2015-2016}

% Examenperiode
%  - 1e semester = 1e examenperiode
%  - 2e semester = 2e examenperiode
%  - tweede zit = 3e examenperiode
\newcommand{\examenperiode}{Tweede examenperiode}

%%========================================================================
%% Inhoud document
%%========================================================================

\begin{document}

%%---------- Front matter ------------------------------------------------
%% Het voorblad - Hier moet je in principe niets wijzigen.

\begin{titlepage}
  \newgeometry{top=2cm,bottom=1.5cm,left=1.5cm,right=1.5cm}
  \begin{center}

    \begingroup
    \rmfamily
    \includegraphics[width=2.5cm]{img/HG-beeldmerk-woordmerk}\\[.5cm]
    \faculteit\\[3cm]
    \titel
    \vfill
    \student\\[3.5cm]
    \rapporttype\\[2cm]
    Promotor:\\
    \promotor\\
    Co-promotor:\\
    \copromotor\\[2.5cm]
    Instelling: \instelling\\[.5cm]
    Academiejaar: \academiejaar\\[.5cm]
    \examenperiode
    \endgroup

  \end{center}
  \restoregeometry
\end{titlepage}

% Schutblad

\emptypage


\begin{titlepage}
  \newgeometry{top=5.35cm,bottom=1.5cm,left=1.5cm,right=1.5cm}
  \begin{center}

    \begingroup
    \rmfamily
    \faculteit\\[3cm]
    \titel
    \vfill
    \student\\[3.5cm]
    \rapporttype\\[2cm]
    Promotor:\\
    \promotor\\
    Co-promotor:\\
    \copromotor\\[2.5cm]
    Instelling: \instelling\\[.5cm]
    Academiejaar: \academiejaar\\[.5cm]
    \examenperiode
    \endgroup

  \end{center}
  \restoregeometry
\end{titlepage}


\begin{abstract}
% TODO: De "abstract" of samenvatting is een kernachtige (max 1 blz. voor een
% thesis) synthese van het document. In ons geval beschrijf je kort de
% probleemstelling en de context, de onderzoeksvragen, de aanpak en de
% resultaten.
\end{abstract}

\chapter*{Voorwoord}
\label{ch:voorwoord}

% TODO: Vergeet ook niet te bedanken wie je geholpen/gesteund/... heeft

\tableofcontents

% Als je een lijst van afkortingen of termen wil toevoegen, dan hoort die
% hier thuis. Gebruik bijvoorbeeld de ``glossaries'' package.

%%---------- Kern --------------------------------------------------------

\chapter{Inleiding}
\label{ch:inleiding}

\section{Probleemstelling en Onderzoeksvragen}
\label{sec:onderzoeksvragen}

% TODO: Wees zo concreet mogelijk bij het formuleren van je
% onderzoeksvra(a)g(en). Een onderzoeksvraag is trouwens iets waar nog
% niemand op dit moment een antwoord heeft (voor zover je kan nagaan).
Angular 2 is nog maar sinds december 2015 in beta, om deze reden is er nog geen overvloed aan informatie beschikbaar. Als developer of bedrijf de opstap maken van Angular 1.x naar Angular 2.0 zal daarom dus niet vanzelfsprekend zijn. Er zijn reeds artikels die een speculatie maken over de migratie hiernaar, vooral in de weken na de aankondiging van Angular 2. Maar echte concrete of goed onderbouwde stappenplannen om een goede overgang te maken zijn er nog niet. Waarom zou een developer overstappen van Angular 1.x naar Angular 2.0, is de huidige Angular niet meer goed genoeg? Wat zijn de belangrijkste veranderingen, welke stappen zal hij/zijn moeten ondergaan om een naadloze overgang te maken en is het eigenlijk zelfs wel nuttig een overstap te maken met wat momenteel geweten is?

\chapter{Methodologie}
\label{ch:methodologie}

% TODO: Hoe ben je te werk gegaan? Verdeel je onderzoek in grote fasen, en
% licht in elke fase toe welke stappen je gevolgd hebt. Verantwoord waarom je
% op deze manier te werk gegaan bent. Je moet kunnen aantonen dat je de best
% mogelijke manier toegepast hebt om een antwoord te vinden op de
% onderzoeksvraag.
\paragraph{Literatuurstudie}
\leavevmode
\newline
Wat is er momenteel geweten over Angular 2?
\begin{itemize}
    	\item Angular developer guide onderzoeken.
	\item Blogs bekijken van ervaren developers (eventueel van Angular team members zoals Victor Savkin, ...).
	\item Inhoudelijk bijstaan door vakexperts: Joeri Van Steen (lector webapplicaties Hogent), developers Array51.
	\item Eventueel interviews regelen met developers. Mogelijke vragen kunnen dan zijn: \newline Zijn er al stappen gezet in het overgaan naar Angular 2, wordt er bij huidige projecten al rekening gehouden met Angular 2, ...
	
\end{itemize}
 
\paragraph{Vergelijking} 
\leavevmode
\newline
Wat zijn de grote verschillen tussen angular 1.x en angular 2? De grootste veranderingen qua functionaliteit/performantie aankaarten en bespreken.
Dit zal ik doen aan de hand van de developer guides van Angular.
\paragraph{Migratieplan} 
\leavevmode
\newline
Aan de hand van alle informatie die ik bij vorige fasen verzameld heb zal ik een goed onderbouwd migratieplan maken. E\'en enkel migratieplan zal echter nooit voldoende zijn omdat er zeer veel verschillende soorten projecten en organisaties zijn. Daarom zal ik ook hiermee rekening moeten houden. Moet je meteen overstappen of kan je in kleinere fasen werken? 


\chapter{Waarom overstappen?}
\label{ch:waarom}

%% TODO: de structuur en titel van deze hoofdstukken hangen af van je
% eigen onderzoek. Elke fase in je onderzoek kan een eigen hoofdstuk krijgen. Kies telkens een gepaste titel. ``Corpus'' is *GEEN* gepaste titel
Angular 1.x werkt zoals in de probleemstelling vermeld werd nog steeds goed en zal de eerstkomende jaren zeker niet verdwijnen. Maar Angular 2 is gemaakt om beter te zijn dan Angular 1. Angular 1 zal de komende jaren dus zeker nog gebruikt kunnen worden om applicaties te ontwikkelen, met als grote reden dat mensen nog steeds oudere technologie\"en gebruiken bijvoorbeeld: IE8, oude versies van Android, \ldots
\newline
\newline
Voor ik overga naar hoe up te graden naar Angular 2 geef ik in dit hoofdstuk enkele redenen waarom Angular 2 het overwegen waard is. 

\paragraph{Perfomantie} 
Verschillende modules zullen verwijderd worden uit de Angular core, wat resulteert in beter performantie. Dit wil niet zeggen dat je deze modules niet meer kan gebruiken, je hoeft nadien enkel de modules die je nodig hebt te zoeken en gebruiken.
\par

\paragraph{Betere mobile support}
Angular 1.x is niet gebouwd met mobile support in het achterhoofd, hoewel er wel enkele frameworks zoals Ionic hier een oplossing voor bieden. Angular 2 heeft uit deze fouten geleerd en meteen frameworks, zoals Ionic, ge\"implementeerd wat een grote inpakt heeft op de user experience en performantie ervan. 
\par

\paragraph{Toekomst} 
Angular 2 gebruikt alle nieuwe, veelbelovende features die momenteel beschikbaar zijn, zoals ES6, Typescript, WebComponents, \ldots
\par


\chapter{Veranderingen in Angular 2.0}
\label{ch:veranderingen}
Een nieuwe versie brengt natuurlijk ook verandering met zich mee. In dit hoofdstuk bespreek ik de grootste veranderen tegenover Angular 1.x.

\paragraph{Typescript}
Typescript is een open source programmeer taal ontwikkeld door Microsoft. Microsoft werkte gedurende enkele maanden samen met het Angular team om Typescript te verbeteren. Resultaat van deze samenwerking was de nieuwe release van de taal, Typescript 1.5. Dit is bijgevolg ook de taal waarin Angular 2 gebouwd is. Hieraan zijn ook enkele voordelen verbonden.  

\begin{itemize}
    	\item Type errors worden compile-time en runtime gegeven. Op die manier is het makkelijker je code te debuggen en te checken of je applicatie werkt.
	\item Betere auto complete door tekst editors en IDEs.
	\item De Javascript VM kan beter code optimizations maken.
\end{itemize}

Typescript is een superset van ES6, wat op zijn beurt een superset van ES5 is. Angular 2 wordt momenteel getranspiled naar ES5 om uitvoerbaar te zijn op moderne browsers. Developers kunnen natuurlijk kiezen in welke taal ze willen werken (ES5, ES6, Typescript, \ldots).
\par

\paragraph{Component based}
Angular 2 is volledig component based, dit wil zeggen dat zaken als controllers en directives niet meer gebruikt zullen worden. De controllers en directives zullen vervangen worden door componenten. Componenten zijn directives met een template die twee zaken bevatten, een selector en een @View. De selector correspondeert met de html tag waarmee de component wil communiceren en de @View specifieert een html template waarin de component gebruikt zal worden.
\par

\newpage
\leavevmode
\newline
Een voorbeeld zal meer duidelijkheid schetsen.
\linebreak
\newline
Angular 1.x:

\begin{lstlisting}[frame=single]
angular.module('voorbeeld')
	.controller('VoorbeeldCtrl', function() {});
\end{lstlisting}

\leavevmode
\newline Angular 2.0:

\begin{lstlisting}[frame=single]
Import  {Component, View} from 'angular/angular2';

@Component({
    selector: 'voorbeeld'
})

@View({
    templateUrl: './components/voorbeeld/voorbeeld.html'
})

Export class Voorbeeld {}
\end{lstlisting}

\paragraph{Dependency injection} 
Hier komen alle verandering omtrent dependency injection.
\par

\paragraph{\$scope} 
Angular 2 zal geen \$scope meet bevatten. In plaats van properties in de scope te binden in onze templates kan je nu meteen naar properties binden van je componenten. 

\begin{lstlisting}[frame=single]
@Component({
  selector: 'sample-app',
  componentServices: [
    NameList
  ]
})
@Template({
  url: './templates/sample-app.html',
  directives: [Foreach]
})
class SampleApp {
  constructor() {
    this.names = NameList.get();
    this.newName = '';
  }
  addName(newname) {
    this.names.push(newname.value);
    newname.value = '';
  }
}
\end{lstlisting}
?Nu kunnen we meteen binden naar ?this.names? in onze template: 

\begin{lstlisting}[frame=single]
...
<ul>
  <li *foreach="#name in names"></li>
</ul>
...
\end{lstlisting}


\chapter{Angular 1.5}
\label{ch:angular1.5}
Voor ik inga op hoe je de overstap naar Angular 2.0 kan maken wil ik het eerst hebben over Angular 1.5 en waarom het handig kan zijn eerst de overstap hiernaar te maken. Angular 1.5 is namelijk gemaakt met Angular 2 in het achterhoofd. De grootste reden achter deze release is de migratie naar Angular 2 makkelijk ter maken. In deze release zitten features die het mogelijk maken voor developers om Angular 1 applicaties te maken die dichter aanleunen bij de structuur van Angular 2.
\linebreak
\newline
De grootste veranderingen zijn vooral gebaseerd rond componenten omdat dit ook een belangrijke verandering is in Angular 2. Maar Angular 1.5 houdt ook enkele andere features in om development te verbeteren, zoals betere support voor ES6 klassen, ngResource, ngRoute, \ldots

\paragraph{Component based applicaties}
\subparagraph{Component directives defini\"eren}
\leavevmode
\newline
Het is nu mogelijk makkelijk een component te defini\"eren met behulp van de module.component helper methode. Een component moet nu enkel een template, enkele bindings en een controller hebben. Voor de overige directive opties zijn er defaults geset. 

\begin{lstlisting}[frame=single]
myModule.component('myComponent', {
  template: '<h1>Hello {{ $ctrl.getFullName() }}</h1>',
  bindings: { firstName: '<', lastName: '<' },
  controller: function() {
    this.getFullName = function() {
      return this.firstName + ' ' + this.lastName;
    };
  }
});

<my-component first-name="'Alan'" last-name="'Rickman'">
</my-component>
\end{lstlisting}


\subparagraph{Lifecycle hooks}
\leavevmode
\newline
Als je component controller een \$onInit() methode heeft zal de compiler deze oproepen wanneer de component ge\"initialiseerd is. Dit zorgt voor een duidelijke plaats om de initialisatie code voor je componenten te plaatsen. Dit is ook gelijkaardig aan de ngOnInit() lifecycle hook van Angular 2.

\paragraph{Binding to required directives}
\leavevmode
\newline
Er kan nu een object gespecificeerd worden voor de require property van directives/componenten. Wanneer gebruikt wordt met bindToController zal de compiler deze required directive controller binden aan je controller. Deze bindings zullen al gebeurd zijn wanneer de \$onInit hook wordt opgeroepen.

\paragraph{Default transclusion content}
\leavevmode
\newline
ngTransclude zal niet langer content overschrijven wanneer er geen content is om op te vullen. Nu kan je bijvoorbeeld default content voorzien wanneer de gebruiker van je directive geen content heeft voorzien om in te voegen.

\paragraph{Multi-slot transclusion}
\paragraph{One-way binding}
\paragraph{ngResource}
\paragraph{ngRoute}

\chapter{Overstappen naar Angular 2.0}
\label{ch:conclusie}
De overstap maken naar Angular 2 in je project kan je op twee manieren aanpakken. De manier die je kiest hangt natuurlijk af van de vereisten van je project. 
\linebreak
\newline
[Misschien ook handig eerst uit te leggen hoe je een Angular 2 app maakt, zonder overgang van Angular 1? https://angular.io/docs/ts/latest/quickstart.html ]

\paragraph{ngForward}
Wanneer je nog niet zeker weet of nog twijfels heb of je je project wilt upgraden naar Angular 2 kan je het framework ngForward gebruiken. Hiermee kan je Angular 1 applicaties maken die de look en feel van Angular 2 hebben. Op die manier kan je toch profiteren van de voordelen van Angular 2 terwijl je in je comfort zone blijft.

\paragraph{ngUpgrade}
Wanneer je meer wilt dan enkel de look van Angular 2 en je klaar bent de overstap te maken naar Angular 2 is het tijd je applicatie te upgraden. Hiervoor kan je ook ngForward gebruiken maar het probleem hierbij is wanneer je een groot Angular 1.x project heb, je ook alles zal moeten herschrijven om te voldoen aan de Angular 2 vereisten. Dit is zeker geen ideale oplossing en daarom is ngUpgrade gebundeld met Angular 2. 




\chapter{Conclusie}
\label{ch:conclusie}

% TODO: Trek een duidelijke conclusie, in de vorm van een antwoord op de
% onderzoeksvra(a)g(en). Reflecteer kritisch over het resultaat. Zijn er
% zaken die nog niet duidelijk zijn? Heeft het ondezoek geleid tot nieuwe
% vragen die uitnodigen tot verder onderzoek?


\bibliographystyle{apa}
\bibliography{tin-bachproef}

%%---------- Back matter -------------------------------------------------

\listoffigures
\listoftables

\end{document}
